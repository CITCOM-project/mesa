\documentclass{article}

\usepackage{a4wide}

\title{Predator-Prey Causal Specification}
\author{Michael Foster}

\begin{document}
\maketitle

\section{Desirable Properties}

\subsection{Two Agent Model}
\begin{enumerate}
  \item If the wolves go extinct, the sheep population will explode since they no longer have any predators to eat them and they don't starve to death as they would in the three agent model.
  \item If the conditions are stupidly optimal wolf and sheep population explosion may occur. This happens if the sheep birth rate massively outstrips the rate they're being eaten at. Once there's a ``critical mass'' of sheep the population can explode unchecked by the wolves. As long as the wolf population growth does not outstrip the supply of sheep, both populations continue to spiral upwards. The wolf birth rate needs to be much lower than the sheep one to support this.
  \item Stability is extremely unlikely. Generally either wolves kill all the sheep and you get total extinction or the wolves die off before all the sheep and you get a sheep explosion after.
\end{enumerate}

\subsection{Three Agent Model}
\begin{enumerate}
  \item Sheep extinction is caused by grass extinction. If the sheep can't find a grass to eat before they run out of energy, they will die. This is affected by grass regrowth time, the number of sheep eating the grass (which is affected between generations by the birth rate), and the amount of energy a sheep gains from eating a grass.
  \item The sheep population can never spiral out of control. This is caused by the fact that the sheep depend on grass for energy, so die off if there's not enough grass to support the current population.
\end{enumerate}

\subsection{Two and Three Agent Model}
\begin{enumerate}
  \item Sheep extinction causes eventual wolf extinction, i.e. the wolves cannot survive (long) without the sheep. The time this takes is affected by the number of wolves and the amount of energy they have (which is affected by the amount of energy a wolf gains from food). The wolves can also go extinct without the sheep going extinct if they run out of energy before they find a sheep to eat. This is affected by the number of sheep and their birth rate.
  \item The more AGENT (wolf or sheep) you have, the more AGENT (wolf or sheep) you get in the next generation.
  \item The sheep population is affected by the wolves - more wolves now leads to less sheep in the future.
  \item If there are wolves and the sheep population increases, the wolf population will also increase.
  \item If the number of wolves outstrips the supply of sheep (i.e. passes the point of equilibrium), the number of sheep will decrease.
  \item If the number of sheep decrease, the wolf population will decrease soon after.
  \item If there are sheep and number of wolves decreases (before the sheep run out of energy), the number of sheep will increase.
\end{enumerate}

\end{document}
